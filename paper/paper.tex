\documentclass[journal]{vgtc}                % final (journal style)
%\documentclass[review,journal]{vgtc}         % review (journal style)
%\documentclass[widereview]{vgtc}             % wide-spaced review
%\documentclass[preprint,journal]{vgtc}       % preprint (journal style)
%\documentclass[electronic,journal]{vgtc}     % electronic version, journal

%% Uncomment one of the lines above depending on where your paper is
%% in the conference process. ``review'' and ``widereview'' are for review
%% submission, ``preprint'' is for pre-publication, and the final version
%% doesn't use a specific qualifier. Further, ``electronic'' includes
%% hyperreferences for more convenient online viewing.

%% Please use one of the ``review'' options in combination with the
%% assigned online id (see below) ONLY if your paper uses a double blind
%% review process. Some conferences, like IEEE Vis and InfoVis, have NOT
%% in the past.

%% Please note that the use of figures other than the optional teaser is not permitted on the first page
%% of the journal version.  Figures should begin on the second page and be
%% in CMYK or Grey scale format, otherwise, colour shifting may occur
%% during the printing process.  Papers submitted with figures other than the optional teaser on the
%% first page will be refused.

%% These three lines bring in essential packages: ``mathptmx'' for Type 1
%% typefaces, ``graphicx'' for inclusion of EPS figures. and ``times''
%% for proper handling of the times font family.

\usepackage{mathptmx}
\usepackage{graphicx}
\usepackage{times}

%% We encourage the use of mathptmx for consistent usage of times font
%% throughout the proceedings. However, if you encounter conflicts
%% with other math-related packages, you may want to disable it.

%% This turns references into clickable hyperlinks.
\usepackage[bookmarks,backref=true,linkcolor=black]{hyperref} %,colorlinks
\hypersetup{
  pdfauthor = {},
  pdftitle = {},
  pdfsubject = {},
  pdfkeywords = {},
  colorlinks=true,
  linkcolor= black,
  citecolor= black,
  pageanchor=true,
  urlcolor = black,
  plainpages = false,
  linktocpage
}

%% If you are submitting a paper to a conference for review with a double
%% blind reviewing process, please replace the value ``0'' below with your
%% OnlineID. Otherwise, you may safely leave it at ``0''.
\onlineid{0}

%% declare the category of your paper, only shown in review mode
\vgtccategory{Research}

%% allow for this line if you want the electronic option to work properly
\vgtcinsertpkg

%% In preprint mode you may define your own headline.
%\preprinttext{To appear in an IEEE VGTC sponsored conference.}

%% Paper title.

\title{Pleiades: Interactive Composing Tools for Vega-Lite Charts}

%% This is how authors are specified in the journal style

%% indicate IEEE Member or Student Member in form indicated below
\author{Chanwut Kittivorawong, Manesh Jhawar, Sorawee Porncharoenwase}
\authorfooter{
%% insert punctuation at end of each item
\item
 Chanwut Kittivorawong is an undergraduate Computer Science student at the University of Washington. E-mail: chanwutk@cs.washington.edu.
\item
 Manesh Jhawar is an undergraduate Computer Science student at the University of Washington. E-mail: mj06@uw.edu.
\item
 Sorawee Porncharoenwase is a graduate Computer Science student at the University of Washington. E-mail: sorawee@cs.washington.edu.
}

%other entries to be set up for journal
%\shortauthortitle{Firstauthor \MakeLowercase{\textit{et al.}}: Paper Title}

%% Abstract section.
\abstract{
[...need better abstract here...]

Since Vega-Lite is a high-level grammar, the grammar is easy to understand,
so the range of users using the software can be from computer science expert to coding
beginner. Although users with less JSON experience can work with most of the Vega-Lite
easily, View Composition requires a good understanding of tree structure since View
Composition can be nested inside each other to create complex views. Beginner users
can struggle using this feature.

This software provides a graphical user interface for users to compose Vega-Lite charts.
The users can add charts that they want to work with to the software, then they can compose
them with layer, concat, repeat, and facet. They can also compose composed chart with one
of the four technique. The software also provide warnings when the users are trying to
compose charts that are incompatible with each other to avoid unexpected behaviors.
} % end of abstract

%% Keywords that describe your work. Will show as 'Index Terms' in journal
%% please capitalize first letter and insert punctuation after last keyword
\keywords{Data Visualization, interactive system, Vega-Lite}

%% ACM Computing Classification System (CCS). 
%% See <http://www.acm.org/class/1998/> for details.
%% The ``\CCScat'' command takes four arguments.

\CCScatlist{ % not used in journal version
 \CCScat{K.6.1}{Management of Computing and Information Systems}%
{Project and People Management}{Life Cycle};
 \CCScat{K.7.m}{The Computing Profession}{Miscellaneous}{Ethics}
}

%% Uncomment below to include a teaser figure.
%   \teaser{
%  \centering
%  \includegraphics[width=16cm]{screenshot.png}
%   \caption{add some picture?}
%   }

%% Uncomment below to disable the manuscript note
%\renewcommand{\manuscriptnotetxt}{}

%% Copyright space is enabled by default as required by guidelines.
%% It is disabled by the 'review' option or via the following command:
% \nocopyrightspace

%%%%%%%%%%%%%%%%%%%%%%%%%%%%%%%%%%%%%%%%%%%%%%%%%%%%%%%%%%%%%%%%
%%%%%%%%%%%%%%%%%%%%%% START OF THE PAPER %%%%%%%%%%%%%%%%%%%%%%
%%%%%%%%%%%%%%%%%%%%%%%%%%%%%%%%%%%%%%%%%%%%%%%%%%%%%%%%%%%%%%%%%

\begin{document}

%% The ``\maketitle'' command must be the first command after the
%% ``\begin{document}'' command. It prepares and prints the title block.

%% the only exception to this rule is the \firstsection command
\firstsection{Introduction}

\maketitle

... intro here

% how to cite
facilisi~\cite{notes2002}

\section{Related Work}

Pleiades is built as a tool to help Vega-Lite users easily create multiple views
visualization by providing a Graphical User Interface to perform View Composition.

\subsection{Vega-Lite: A Grammar of Interactive Graphics}

...about Vega-Lite relating to this project...
% how to do foot note
\footnote{how to do foot note}.

\section{Methods}
\subsection{User Interface}

To work with Pleiades, users can add Vega-Lite specs that they are working with
to the left sidebar by clicking "NEW SPEC", then type in the Vega-Lite spec, and
save.

To create view composition, users can select view(s) as operand(s) and then apply
an operation. It will output the composed view in the main view area. Users are
allowed to select up to one view from the sidebar and up to one view from the main
view to perform an operation. Users can also perform an operation to composed views.
For example, a layered view can then be horizontal concatenated with another view.
Then, inside the concatenated views, the right view can be selected to repeat.

To edit composed view, users can select a view in the main view. Then, click "EDIT"
configure properties of the selected view.

Pleiades also perform minor validation before committing to any operation to prevent
an unexpected output. For example, when the user selects operands to layer, if one
of the operands is not a unit spec or layer spec, the "LAYER" button is disabled.
And if both operands do not have compatible axes, the "LAYER" button will show a
warning sign with tooltip that the axes are not compatible.

Pleiades also provides Inner View Navigator that shows the view from inner spec
of repeat or facet view. Repeat and facet operation produces a view containing
replication of the operand. When selecting repeat or facet view, the Inner View
Navigator shows the original view before the replication. This is useful when
the original view is also a composite view. Then, we can select the inner view
to edit from the Inner View Navigator.

Finally when the user is done with composing view, they can export to Vega-Lite
json file to use normally with any Vega-Lite compiler.

\subsection{Operations for View Composition}
There are 5 main operations users can do to compose views

\subsubsection{Place}
When the main view is empty, user can select a view in the sidebar. Then click
"PLACE" to place the view to the main view.

\subsubsection{Layer}


% how to do paragraph
\paragraph{how to add paragraph}

this is how to add pagrgraph

%% if specified like this the section will be committed in review mode
\acknowledgments{
The authors wish to thank A, B, C. This work was supported in part by
a grant from XYZ.}

\bibliographystyle{abbrv}
%%use following if all content of bibtex file should be shown
%\nocite{*}
\bibliography{template}
\end{document}
